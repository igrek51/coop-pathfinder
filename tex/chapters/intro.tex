\chapter{Wstęp}
\label{ch:wstep}

\section{Cel i zakres pracy}
Przedmiotem niniejszej pracy jest przegląd metod wykorzystywanych do planowania bezkolizyjnych tras dla wielu robotów mobilnych.
Stanowi to również wstęp teoretyczny do zaprojektowania algorytmu i implementacji oprogramowania pozwalającego na symulację działania skutecznego planowania tras dla systemu wielorobotowego.

Praca skupia się na przypadkach, w których mamy do czynienia ze środowiskiem z dużą liczbą przeszkód (np. zamknięty budynek z licznymi ciasnymi korytarzami), aby uwypuklić  problem blokowania sią agentów często prowadzący do zakleszczenia. Często okazuje się, że należy wtedy zastosować nieco inne podejścia niż te, które sprawdzają się w przypadku otwartych środowisk, a które zostały opisane np. w pracach \cite{roszkowska}, \cite{siemiatkowska}.
W otwartych środowiskach z małą liczbą przeszkód wystarczające może się okazać np. proste replanowanie wykorzystujące algorytm D* (por. \ref{ch:dstar}) lub LRA* (por. \ref{ch:lra}).

W niniejszej pracy starano się znaleźć metody rozwiązujące zagadnienie, w którym znane są:
\vspace{-1em}
\begin{itemize}[noitemsep]
	\item pełna informacja o mapie otoczenia (położenie statycznych przeszkód),
	\item aktualne położenie i położenie celu dla każdego z robotów.
\end{itemize}

Szukany jest natomiast przebieg tras do punktów docelowych dla agentów. Zadaniem algorytmu będzie wyznaczenie możliwie najkrótszej bezkolizyjnej trasy dla wszystkich robotów. Należy jednak zaznaczyć, że priorytetem jest dotarcie każdego z robotów do celu bez kolizji z innymi robotami. Drugorzędne zaś jest, aby wyznaczone drogi były możliwie jak najkrótsze.

\clearpage
\subsection{Założenia}
Założenia do rozważanego problemu:
\begin{enumerate}
	\item Każdy z robotów ma wyznaczony inny punkt docelowy, do którego zmierza.
	\item Planowanie tras dotyczy robotów holonomicznych.
	\item Czas trwania zmiany kierunku robota jest pomijalnie mały.
	\item Srodowisko zawiera dużą liczbę przeszkód oraz wąskie korytarze (por. rys. \ref{fig:img_robopath_sample-maze}).
	\item Roboty "wiedzą" o sobie i mogą komunikować się ze sobą podczas planowania tras.
	\item Każdy robot zajmuje w przestrzeni jedno pole. Na jednym polu może znajdować się maksymalnie jeden robot.
	\item Planowanie tras powinno odbywać się w czasie rzeczywistym.
\end{enumerate}

\begin{figure}[H]
	\centering
	\includegraphics[width=9cm]{img/robopath/sample-maze}
	\caption{Przykładowe środowisko z dużą liczbą przeszkód i rozmieszczonymi robotami. Źródło: własna implementacja oprogramowania symulacyjnego}
	\label{fig:img_robopath_sample-maze}
\end{figure}


\clearpage
\section{Koordynacja ruchu robotów}
Koordynacja ruchu robotów jest jednym z fundamentalnych problemów w systemach wielorobotowych. \cite{optpriorities}

Kooperacyjne znajdowanie tras (ang. {\it Cooperative Pathfinding}) jest zagadnieniem planowania w układzie wieloagentowym, w którym to agenci mają za zadanie znaleźć bezkolizyjne drogi do swoich, osobnych celów. Planowanie to odbywa się w oparciu o pełną informację o środowisku oraz o trasach pozostałych agentów. \cite{cooppath}

Algorytmy do wyznaczania bezkolizyjnych tras dla wielu agentów (robotów) mogą znaleźć zastosowanie w szpitalach (np. roboty TUG i HOMER do dostarczania sprzętu na wyposażeniu szpitala \cite{tughomer}) oraz magazynach (np. roboty transportowe w magazynach firmy Amazon \ref{fig:image_kiva_amazon}).

\begin{figure}[H]
	\centering
	\includegraphics[width=14cm]{img/kiva-amazon}
	\caption{Roboty Kiva pracujące w magazynie firmy Amazon. Źródło: \cite{amazonkiva}}
	\label{fig:image_kiva_amazon}
\end{figure}

\subsection{Podobieństwo do gier RTS}
Problem kooperacyjnego znajdowania tras pojawia się nie tylko w robotyce, ale jest również popularny m.in. w grach komputerowych, gdzie konieczne jest wyznaczanie bezkolizyjnych dróg dla wielu jednostek, unikając wzajemnego blokowania się. Brak wydajnych i skutecznych algorytmów planowania dróg można zauważyć w wielu grach typu RTS ({\it Real-Time Strategy games}), gdzie czasami obserwuje się zjawisko zakleszczenia jednostek w wąskich gardłach (np. Age of Empires II, Warcraft III lub we współczesnych produkcjach) \cite{efficient_coop_pathplanning} (por. rys. \ref{fig:img_games_age-deadlock}). Ponadto, brak ogólnie dostępnych implementacji lub bibliotek do rozwiązania problemu typu {\it Cooperative Pathfinding} świadczy o potrzebie rozwoju tych metod.

Często algorytmy wykorzystywane w grach typu RTS zajmują się planowaniem dróg dla układu wielu agentów w czasie rzeczywistym (będącego przedmiotem niniejszej pracy), dlatego można stosować je zamiennie również do koordynacji ruchu zespołu robotów.

$TODO$ zrobić screen z pożądnym zakleszczeniem jednostek
\begin{figure}[H]
	\centering
	\includegraphics[width=14cm]{img/games/age-deadlock}
	\caption{Popularny problem zakleszczania się jednostek. Źródło: gra komputerowa Age of Empires II HD}
	\label{fig:img_games_age-deadlock}
\end{figure}

\section{Podstawowe pojęcia}
\subsubsection{Robot holonomiczny}
Robot holonomiczny to taki robot mobilny, który może zmienić swoją orientację, stojąc w miejscu.

\subsubsection{Przestrzeń konfiguracyjna}
Przestrzeń konfiguracyjna to N-wymiarowa przestrzeń będąca zbiorem możliwych stanów danego układu fizycznego.
Wymiar przestrzeni zależy od rodzaju i liczby wyróżnionych parametrów stanu.
W odróżnieniu od przestrzeni roboczej, gdzie robot ma postać bryły, w przestrzeni konfiguracyjnej robot jest reprezentowany jako punkt.

\subsubsection{Zupełność algorytmu (ang. \it Completeness)}
W kontekście algorytmu przeszukiwania grafu algorytm zupełny to taki, który gwarantuje znalezienie rozwiązania, jeśli takie istnieje.
Nie gwarantuje natomiast, że znalezione rozwiązanie będzie rozwiązaniem optymalnym.

\subsubsection{Metoda hill-climbing}
Metoda hill-climbing jest rodzajem matematycznej optymalizacji, lokalną metodą przeszukiwania.
Jest to iteracyjny algorytm, który zaczyna w wybranym rozwiązaniu problemu, następnie próbuje znaleźć lepsze rozwiązanie poprzez przyrostowe zmiany pojedynczych elementów rozwiązania.
Jeśli zmiana przynosi lepsze rozwiązanie, przyrostowa zmiana jest wprowadzana do nowego rozwiązania.
Kroki algorytmu powtarzane są dotąd, aż żadna zmiana nie przynosi poprawy.


\chapter{Metody planowania tras}
\label{ch:path_planning_methods}

\section{Metody planowania tras}
$TODO$ wyjebać to gówno lub uogólnić
Spośród metod wykorzystywanych do planowania tras dla wielu robotów można wyróżnić dwie zasadnicze grupy \cite{latombe}:
$TODO$ zcentralizowanie nie muszą być optymalne
\begin{itemize}
	\item {\bf Zcentralizowane} - drogi wyznaczane są dla wszystkich agentów na raz (jednocześnie). Metody te potrafią znaleźć wszystkie możliwe rozwiązania (w szczególności to optymalne), ale mają bardzo dużą złożoność obliczeniową ze względu na ogromną przestrzeń przeszukiwania. Z tego powodu stosowane są heurystyki przyspieszające proces obliczania rozwiązania.
	\item {\bf Rozproszone} (ang. {\it decoupled} lub {\it distributed}) - Podejście to dekomponuje zadanie na niezależne lub zależne w niewielkim stopniu problemy dla każdego agenta. Dla każdego robota droga wyznaczana jest osobno, w określonej kolejności, następnie rozwiązywane są konflikty (kolizje dróg). W pewnych przypadkach rozwiązanie może nie zostać znalezione, mimo, iż istnieje. Zastosowanie metod rozproszonych wiąże się najczęściej z koniecznością przydzielenia priorytetów robotom, co stanowi istotny problem, gdyż od ich wyboru może zależeć zupełność algorytmu. Nie należy mylić tej metody z zagadnieniem typu {\it Non-Cooperative Pathfinding}, w którym agenci nie mają wiedzy na temat pozostałych planów i muszą przewidywać przyszłe ruchy pozostałych robotów \cite{cooppath}. W podejściu rozproszonym agenci mają pełną informację na temat stanu pozostałych robotów, lecz wyznaczanie dróg odbywa się w określonej kolejności.
\end{itemize}

$TODO$
- Plan all units simultaneously
	- Computationally intractable
	- $(units x actions)^depth$
- Plan individual units
	- Not complete
	- A lot of techniques needed to be practical

$TODO$
Dyskretna siatka pól - podział w celu szybszych obliczeń i wykorzystania A* (większość jest oparta na tym) - tak się stosuje w grach
$TODO$ - tutaj screen z warcraft map editora
Obrazek: Transform terrain to large grid [screenshot taken from Warcraft III map editor]. \cite{hierpathfindinginrts}
Use abstract maps to reduce computational costs
$TODO$ Pacman tu albo przy heurystykach
\begin{figure}[H]
	\centering
	\includegraphics[width=13cm]{img/paclan1}
	\caption{Podział mapy na siatkę pól, zastosowanie heurystyki uwzględniającej zawijanie mapy po bokach, Źródło: własna implementacja gry PacMan}
	\label{fig:image_paclan1}
\end{figure}

\subsection{Metody zcentralizowane}
Wiele metod zcentralizowanych cechuje się planowaniem w zbiorowej przestrzeni konfiguracyjnej oraz możliwością wyznaczenia optymalnego rozwiązania.
Wadą jest natomiast duża złożoność obliczeniowa algorytmu i konieczność posiadania pełnej informacji o stanie otoczenia i robotów.

W systemach czasu rzeczywistego istotne jest, aby rozwiązanie problemu planowania tras uzyskać w określonym czasie, dlatego w tego typu sytemach częściej używane są techniki rozproszone.

$TODO$ Podejmowanie decyzji na podstawie centralnego systemu - scentralizowana struktura organizacyjna

\subsection{Metoda pól potencjałowych}
$TODO$ nie pisać, że to zcentralizowana, czy nie
Nie wszystkie podejścia zcentralizowane gwarantują optymalne rozwiązanie. Przykładem takiej metody, która nie daje gwarancji optymalności (ani nawet zupełności) jest metoda pól potencjałowych.

Metoda pól potencjałowych (ang. {\it Artificial Potential Field} lub {\it Potential Field Techniques}) polega na zastosowaniu zasad oddziaływania między ładunkami znanych z elektrostatyki. Roboty i przeszkody traktowane są jako ładunki jednoimienne, przez co "odpychają się" siłą odwrotnie proporcjonalną do kwadratu odległości (dzięki temu unikają kolizji między sobą). Natomiast punkt docelowy robota jest odwzorowany jako ładunek o przeciwnym biegunie, przez co robot jest "przyciągany" do celu.
Główną zasadę działania metody przedstawiono na rysunku \ref{fig:image_potentialfield}.
Technika ta jest bardzo prosta i nie wymaga wykonywania złożonych obliczeń (w odróżnieniu do pozostałych metod zcentralizowanych). Niestety bardzo powszechny jest problem osiągania minimum lokalnego, w którym suma wektorów daje zerową siłę wypadkową. Robot zostaje "uwięziony" w minimum lokalnym, przez co nie jest w stanie dotrzeć do wyznaczonego celu. Do omijania tego problemu muszą być stosowane inne dodatkowe metody. \cite{potentialfield}
\begin{figure}[H]
	\centering
	\includegraphics[width=12cm]{img/potential-field}
	\caption{Zasada działania metody pól potencjałowych. Źródło: \cite{howie_potentialfield}}
	\label{fig:image_potentialfield}
\end{figure}

\subsection{Rozproszone wyznaczanie tras}
$TODO$ wyjebać to gówno
Popularne podejścia unikające planowania w wysoko wymiarowej zbiorowej przestrzeni konfiguracyjnej to techniki rozproszone i priorytetowane.
Pomimo, że metody te są bardzo efektywne, mają dwie główne wady:
\begin{enumerate}
	\item Nie są zupełne - czasami nie udaje się znaleźć rozwiązania, nawet gdy istnieje.
	\item Wynikowe rozwiązania są często nieoptymalne.
\end{enumerate}

W artykule \cite{optpriorities} przedstawione zostało podejście do optymalizowania układu priorytetów dla rozproszonych i priorytetowanych technik planowania.
Proponowana metoda wykonuje randomizowane przeszukiwanie z techniką hill-climbing do znalezienia rozwiązania i do skrócenia całkowitej długości ścieżek.
Technika ta została zaimplemenotwana i przetestowana na prawdziwych robotach oraz w rozległych testach symulacyjnych.
Wyniki eksperyentu pokazały, że metoda potrafi znacząco zmniejszyć liczbę niepowodzeń i znacznie zmniejszyć całkowitą długość tras dla różnych priorytetowanych i rozproszoncyh metod planowania dróg, nawet dla dużych zespołów robotów.

Najczęściej stosowanymi podejściami są metody oparte o algorytm A*. W szczególności są to:
\begin{itemize}
	\item A* w konfiguracji czaso-przestrzennej
	\item Path coordination
\end{itemize}

\subsubsection{Path coordination}
Idea metody Path coordination przedstawia się następująco \cite{optpriorities}:
\begin{enumerate}
	\item Wyznaczenie ścieżki dla każdego robota {\bf niezależnie}
	\item Przydział priorytetów
	\item Próba rozwiązania możliwych konfliktów między ścieżkami. Roboty utrzymywane są na ich indywidualnych ścieżkach (wyznaczonych na początku), wprowadzane modyfikacje pozwalają na zatrzymanie się, ruch naprzód, a nawet cofanie się, ale tylko {\bf wzdłuż trajektorii} w celu uniknięcia kolizji z robotem o wyższym priorytecie.
\end{enumerate}
Złożoność metody wynosi $O(n \cdot m \cdot log(m))$, $m$ - maksymalna liczba stanów podczas planowania, $n$ - liczba robotów


\subsection{Priorytetowane planowanie}
$TODO$
In general, the complexity of complete approaches to multi-agent
path planning grows exponentially with the number of agents. There-
fore, the complete approaches often do not scale-up well and hence
are often not applicable for nontrivial domains with many agents. To
plan paths for a high number of agents in a complex environment,
one has to resort to one of the incomplete, but fast approaches. A
simple method often used in practice is prioritized planning [3, 9, 1].
In prioritized planning the agents are assigned a unique priority. In
its simplest form, the algorithm proceeds sequentially and agents
plan individually from the highest priority agent to the lowest one.
The agents consider the trajectories of higher priority agents as con-
straints (moving obstacles), which they need to avoid. It is straight-
forward to see that when the algorithm finishes, each agent is as-
signed a trajectory not colliding with either higher priority agents,
since the agent avoided a collision with those, nor with lower priority
agents who avoided a conflict with the given trajectory themselves.

The complexity of the generic algorithm grows linearly with the
number of agents, which makes the approach applicable for problems
involving many agents. Clearly, the algorithm is greedy and incom-
plete in the sense that agents are satisfied with the first trajectory not
colliding with higher priority agents and if a single agent is unable to
find a collision-free path for itself, the overall path finding algorithm
fails. The benefit, however, is fast runtime in relatively uncluttered
environments, which is often the case in multi-robotic applications.
Prioritized planner is also sensitive to the initial prioritization of the
agents. Both phenomena are illustrated in Figure 1 that shows a sim-
ple scenario with two agents desiring to move from s1 to d1 (s2 to d2
resp.) in a corridor that is only slightly wider than a single agent. The
scenario assumes that both agents have identical maximum speeds.
\cite{async_decentralized_spacetime_cp}


\begin{figure}[H]
	\centering
	\includegraphics[width=10cm]{img/prioritized-planning-problem1}
	\caption{Top: example of a problem to which a prioritized planner
will not find a solution. The first agent plans its optimal path first,
but such a trajectory is in conflict with all feasible trajectories of the
second agent. Bottom: example of a problem to which a prioritized
planner will find a solution only if agent 1 has a higher priority than
agent 2. Źródło: \cite{async_decentralized_spacetime_cp}}
	\label{fig:img_prioritized-planning-problem1}
\end{figure}

\subsection{Wybór priorytetów}
$TODO$ nie dotyczy tylko Path coordination, ale też A* time-space
Istotną rolę doboru priorytetów robotów w procesie planowania tras ukazuje prosty przykład przedstawiony na rysunku \ref{fig:image_article1_fig1}. Jeśli robot 1 (zmierzający z punktu S1 do G1) otrzyma wyższy priorytet niż robot 2 (zmierzający z S2 do G2), spowoduje to zablokowanie przejazdu dla robota 2 i w efekcie prawidłowe rozwiązanie nie zostanie znalezione.
\begin{figure}[H]
	\centering
	\includegraphics[width=10cm]{img/article1/fig1}
	\caption{Sytuacja, w której żadne rozwiązanie nie zostanie znalezione, jeśli robot 1 ma wyższy priorytet niż robot 2. Źródło: \cite{optpriorities}}
	\label{fig:image_article1_fig1}
\end{figure}

Układ priorytetów może mieć również zasadniczy wpływ na długość uzyskanych tras. Odpowiedni przykład został przedstawiony na rysunku \ref{fig:image_article1_ppt6}. W zależności od wyboru priorytetów, wpływających na kolejność planowania tras, otrzymujemy różne rozwiązania.
\begin{figure}[H]
	\centering
	\includegraphics[width=13cm]{img/article1/ppt6}
	\caption{a) Niezależne planowanie optymalnych tras dla 2 robotów; b) suboptymalne rozwiązanie, gdy robot 1 ma wyższy priorytet; c) rozwiązanie, gdy robot 2 ma wyższy priorytet}
	\label{fig:image_article1_ppt6}
\end{figure}
