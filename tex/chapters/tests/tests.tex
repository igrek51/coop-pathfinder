\chapter{Wyniki testów}
\label{ch:tests}

\section{Obszerne testy aplikacji}
$TODO$ obszerne testy, porównanie metod: LRA*, WHCA* przy tych samych warunkach początkowych, porównanie czasu wykonania, porównanie tego samego algorytmu w zależności od parametru (np. okna czasowego); badanie skuteczności, długości tras, czasu wykonania, 
do przeprowadzenia testów wykorzystano biblitekę jUnit, która co prawda służy do wykonywania testów jednostkowych sprawdzających poprawność pojedynczych komponentów aplikacji
metodyka przeprowadzenia testów
typy środowisk testowych (rozmiar, roboty, screeny):
	1. 11x11, 5 robotów
	2. 11x11, 10 robotów
	3. 35x35, 5 robotów
	4. 11x11 (bez labiryntu), 30 robotów
histogram liczby kroków potrzebnych do rozwiązania
potwierdzenie oczekiwań (poprawności) - nigdy nie było tak, żeby LRA był lepszy
zwykłego A* nawet nie warto testować
potential field - nawet nie warto testować, raczej jako ciekawostka, nie potrafi doprowadzić do celu nawet jednego robota
przeprowadzenie testów jest trudne i wymaga losowania środowisk i warunków i wyciągania statystyki

\section{Screeny}
$TODO$ screeny ciekawych przypadków

\begin{figure}
	\centering
	\includegraphics[width=0.8\columnwidth]{img/robopath/field-potential-hole}
	\caption{Robot uwięziony w studni potencjału. Zerowa siła wypadkowa nie pozwala mu dotrzeć do celu.}
	\label{fig:app-tech-intellij}
\end{figure}

\begin{figure}
	\centering
	\includegraphics[width=0.8\columnwidth]{img/robopath/lra-bigmap}
	\caption{Metoda LRA*: duża mapa z dużą liczbą robotów}
	\label{fig:app-tech-intellij}
\end{figure}

\begin{figure}
	\centering
	\includegraphics[width=0.8\columnwidth]{img/robopath/lra-cycle}
	\caption{Metoda LRA*: 2 roboty w cyklu akcji}
	\label{fig:app-tech-intellij}
\end{figure}

\begin{figure}
	\centering
	\includegraphics[width=0.8\columnwidth]{img/robopath/lra-lot-robots}
	\caption{Metoda LRA*: dużo robotów, mała mapa}
	\label{fig:app-tech-intellij}
\end{figure}

\begin{figure}
	\centering
	\includegraphics[width=0.8\columnwidth]{img/robopath/puzzle-15}
	\caption{Metoda WHCA*: puzzle 15}
	\label{fig:app-tech-intellij}
\end{figure}

\section{Środowiska}
labirynt, otwarte, puzzle 15

filmiki można obejrzć na YT, link

LRA: wyszło całkiem nieźle,
testy ograniczać trzeba liczbą kroków symulacji, bo LRA może trwać wiecznie

\section{Porównanie wyników}
jak często zwykły A* to za mało - ile razy pojawia się chociaż jedna kolizja?
porównanie WHCA* przy różnych oknach czasowych
porównanie metod przydziału i zmiany priorytetów - jak zmienia się skuteczność  po wprowadzeniu zmiany priorytetów
porównanie LRA* z WHCA*
porównanie CA* z WHCA*
porównanie z potential fields
porównanie WHCA z promocją priorytetów (własną, autorską) i bez + z/bez rozszerzania okna czasowego

\section{Dyskusja wyników}
wolno działa WHCA* przy dużych mapach / oknach czasu / robotach. Dałoby się zoptymalizować (RRA)
autorski WHCA* działa dobrze nawet przy rozwiązywaniu dużych deadlocków, problem z puzzle 15
