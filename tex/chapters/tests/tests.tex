\chapter{Wyniki testów}
\label{ch:tests}

\section{Obszerne testy aplikacji}
$TODO$ obszerne testy, porównanie metod: LRA*, WHCA* przy tych samych warunkach początkowych, porównanie czasu wykonania, porównanie tego samego algorytmu w zależności od parametru (np. okna czasowego); badanie skuteczności, długości tras, czasu wykonania, 
do przeprowadzenia testów wykorzystano biblitekę jUnit, która co prawda służy do wykonywania testów jednostkowych sprawdzających poprawność pojedynczych komponentów aplikacji
nie działa wycofywanie się obu robotów (zawsze jeden czeka)
metodyka przeprowadzenia testów
3 typy środowisk testowych (rozmiar, roboty, screeny)
histogram liczby kroków potrzebnych do rozwiązania
potwierdzenie oczekiwań (poprawności) - nigdy nie było tak, żeby LRA był lepszy
zwykłego A* nawet nie warto testować
potential field - nawet nie warto testować, raczej jako ciekawostka, nie potrafi doprowadzić do celu nawet jednego robota

\section{Screeny}
$TODO$ screeny ciekawych przypadków

\begin{figure}
	\centering
	\includegraphics[width=0.8\columnwidth]{img/robopath/field-potential-hole}
	\caption{Robot uwięziony w studni potencjału. Zerowa siła wypadkowa nie pozwala mu dotrzeć do celu.}
	\label{fig:app-tech-intellij}
\end{figure}

\begin{figure}
	\centering
	\includegraphics[width=0.8\columnwidth]{img/robopath/lra-bigmap}
	\caption{Metoda LRA*: duża mapa z dużą liczbą robotów}
	\label{fig:app-tech-intellij}
\end{figure}

\begin{figure}
	\centering
	\includegraphics[width=0.8\columnwidth]{img/robopath/lra-cycle}
	\caption{Metoda LRA*: 2 roboty w cyklu akcji}
	\label{fig:app-tech-intellij}
\end{figure}

\begin{figure}
	\centering
	\includegraphics[width=0.8\columnwidth]{img/robopath/lra-lot-robots}
	\caption{Metoda LRA*: dużo robotów, mała mapa}
	\label{fig:app-tech-intellij}
\end{figure}

\begin{figure}
	\centering
	\includegraphics[width=0.8\columnwidth]{img/robopath/puzzle-15}
	\caption{Metoda WHCA*: puzzle 15}
	\label{fig:app-tech-intellij}
\end{figure}

\section{Środowiska}
labirynt, otwarte, puzzle 15

\section{Porównanie wyników}
porównanie WHCA* przy różnych oknach czasowych
porównanie metod przydziału i zmiany priorytetów - jak zmienia się skuteczność  po wprowadzeniu zmiany priorytetów
porównanie LRA* z WHCA*
porównanie CA* z WHCA*
porównanie z potential fields
porównanie WHCA2 i kilavuz WHCA ?