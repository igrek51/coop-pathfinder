\section{Wykorzystane technologie}
Do stworzenia aplikacji wykorzystano wiele technologii oraz narzędzi deweloperskich opisanych w poniższych podrozdziałach.

\subsection{Java 8}
Aplikacja w całości została napisana w języku Java.
Java jest językiem programowania ogólnego przeznaczenia, zorientowanym obiektowo. Posiada obsługę wyjątków oraz naturalne wsparcie dla wielowątkowości, pozwala na szybkie tworzenie wysokopoziomowych aplikacji. Aplikacje napisane w języku Java kompilowane są do kodu bajtowego, który uruchamiany jest przez wirtualną maszynę Javy (JVM), niezależnie od architektury urządzenia.

Dzięki wieloplatformowości stworzona aplikacja może być uruchamiana na każdym systmie operacyjnym z wirtualną maszyną Javy.

W projekcie aplikacji wykorzystano język Java w wersji 8. 
Pozwoliło to na użycie w projekcie takich elementów języka, jak: wyrażenia lambda i interfejsy funkcyjne (skracające zapis i zwiększające czytelność kodu) oraz strumieni (stream API) - do szybkiego i przejrzystego przekształcania i filtracji danych.

\subsection{JavaFX}
JavaFX jest platformą do tworzenia aplikacji desktopowych (okienkowych) pisanych w języku Java.
Od wersji 8 JavaFX została właczona do platformy Java Standard Edition.
Jest to nowsze, obecnie zalecane przez firmę {\it Oracle} rozwiązanie do tworzenia aplikacji okienkowych. Nie zaleca się natomiast korzystania z bibliotek {\it AWT} lub {\it Swing} \ref{javafx-replacing-swing}.

Do zbudowania widoków aplikacji wykorzystano specjalny format plików FXML, w którym to JavaFX przechowuje informację o właściwościach komponentów interfejsu użytkownika oraz o ich wzajemnych relacjach.

\subsection{Spring}
\subsubsection{Spring Framework}
Spring jest frameworkiem, który umożliwia w aplikacjach Javy zastosowanie wzorca architektonicznego odwrócenia sterowania (ang. {\it Inversion of Control}), a w szczególności wstrzykiwania zależności (ang. {\it Dependency Injection}). Pozwala to uniknąć występowania bezpośrednich zależności pomiędzy komponentami oraz umożliwia automatyczne dostarczanie i zarządzanie cyklem życia komponentów aplikacji.

W zaprojektowanej aplikacji Spring jest wykorzystywany m.in. do automatycznego dostarczania współdzielonych danych o parametrach symulacji oraz do zarządzania cyklem życia kontrolerów i widoków z biblioteki JavaFX.

\subsubsection{Spring Boot}

\subsubsection{Spring Boot JavaFx Support}

\subsection{jUnit}
testy jednostkowe jUnit
$TODO$ TDD - Test driven development, testy jednostkowe do algorytmów pathfinding

\subsection{Maven}
uruchomienie aplikacji z kodów źródłowych : mvn spring-boot:run
do uruchomienia aplikacji potrzebna jest zainstalowana java i maven

\subsection{IntelliJ IDEA}
IntelliJ Ultimate

Na Linux: Arch i Debian - nie ma to znaczenia

\subsection{Pozostałe narzędzia i biblioteki}
\subsubsection{Guava}
joiner
\subsubsection{git}
\subsubsection{logback}


\section{Struktura aplikacji}
$TODO$ lista beanów / serwisów, struktura widok, prezenter, kontroler; osobny wątek w tle do obliczeń + synchronizacja, wątek UI - zapewnienie REal-time, prawie MVP + MVC


\section{Screeny}
$TODO$ screeny

\section{featurey}
ustawianie random seeda
ponowne wykonanie symulacji - te same warunki
przełaczanie między metodami na zakładkach ?
wykonanie symulacji w pojedynczych krokach
resizable window - responsive
heuristic cache
działania na wektorach, immutable vector

\section{Ograniczenia}
$TODO$ nałożone uproszczenia: ruch skośny trwa tyle samo, czas dyskretny, brak czasu na obrót

$TODO$ publikacja na GitHub, licencja MIT, filmiki na YT