\chapter{Wstęp teoretyczny}
\label{ch:theory}

\section{Podstawowe pojęcia}
\subsubsection{Robot holonomiczny}
Robot holonomiczny to taki robot mobilny, który może zmienić swoją orientację, stojąc w miejscu.

\subsubsection{Przestrzeń konfiguracyjna}
Przestrzeń konfiguracyjna to $N$-wymiarowa przestrzeń będąca zbiorem możliwych stanów danego układu fizycznego.
Wymiar przestrzeni zależy od rodzaju i liczby wyróżnionych parametrów stanu.
W odróżnieniu od przestrzeni roboczej, gdzie robot ma postać bryły, w przestrzeni konfiguracyjnej robot jest reprezentowany jako punkt.

\subsubsection{Zupełność algorytmu (ang. {\it Completeness})}
W kontekście algorytmu przeszukiwania grafu algorytm zupełny to taki, który gwarantuje znalezienie rozwiązania, jeśli takie istnieje.
Warto zaznaczyć, że nie gwarantuje to wcale, że znalezione rozwiązanie będzie rozwiązaniem optymalnym.

% \subsubsection{Metoda hill-climbing}
% Metoda hill-climbing jest rodzajem matematycznej optymalizacji, lokalną metodą przeszukiwania.
% Jest to iteracyjny algorytm, który zaczyna w wybranym rozwiązaniu problemu, następnie próbuje znaleźć lepsze rozwiązanie poprzez przyrostowe zmiany pojedynczych elementów rozwiązania.
% Jeśli przyrostowa zmiana przynosi lepsze rozwiązanie, jest ona wprowadzana do nowego rozwiązania.
% Kroki algorytmu powtarzane są dotąd, aż żadna zmiana nie przynosi już poprawy.
