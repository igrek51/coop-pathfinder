\chapter{Algorytmy oparte o A*}
\label{ch:astar}

Kiedy pojedynczy agent staje przed zadaniem znalezienia drogi do wyznaczonego celu, prosty algorytm A* sprawdza się znakomicie. Jednak w przypadku, gdy wiele agentów porusza się w tym samym czasie, podejście to może się już nie sprawdzić, powodując wzajemne blokowanie się i zakleszczenie jednostek. Rozwiązaniem tego problemu jest kooperacyjne znajdowanie tras. Dzięki tej technice roboty będą mogły skutecznie przemieszczać się przez mapę, omijając trasy wyznaczone przez inne jednostki oraz schodząc innym jednostkom z drogi, jeśli to konieczne \cite{cooppath}.

Zagadnienie znajdowania drogi jest również ważnym elementem sztucznej inteligencji zaimplementowanej w wielu grach komputerowych. Chociaż klasyczny algorytm A* potrafi doprowadzić pojedynczego agenta do celu, to jednak dla wielu agentów wymagane jest zastosowanie innego podejścia w celu unikania kolizji. Istniejące rozwiązania są jednak wciąż oparte o Algorytm A*, choć nieco zmodyfikowany. Chociaż A* może zostać zaadaptowany do ponownego planowania trasy na żądanie, w przypadku wykrycia kolizji tras, to jednak takie podejście nie jest zadowalające pod wieloma względami. Na trudnych mapach z wieloma wąskimi korytarzami i wieloma agentami może to prowadzić do zakleszczenia agentów w wąskich gardłach lub do cyklicznego zapętlenia akcji agentów \cite{cooppath}.

Rozdział ten zmierza do zaprezentowania zasady działania oraz cech algorytmu WHCA* (por. \ref{ch:whca}), zaproponowanego przez Davida Silvera \cite{cooppath}. Zaczynamy jednak od przedstawienia samego algorytmu A* i wprowadzając kolejne modyfikacje (CA*, HCA*), przekształcimy go stopniowo w WHCA*.
W rozdziale zaprezentowano także rozwiązania alternatywne takie, jak: D* (por. \ref{ch:dstar}) i LRA* (por. \ref{ch:lra}).

\section{Algorytm A*}
\label{ch:astar-theory}
\subsection{Zasada działania}
A* jest algorytmem heurystycznym służącym do przeszukiwania grafu w celu znalezienia najkrótszej ścieżki między węzłem początkowym a węzłem docelowym. Algorytm ten jest powszechnie stosowany w zagadnieniach sztucznej inteligencji oraz w grach komputerowych \cite{mit_astar}. Opiera się na zapisywaniu węzłów w dwóch listach: zamkniętych (odwiedzonych) i otwartych (do odwiedzenia). Jest modyfikacją algorytmu Dijkstry poprzez wprowadzenie pojęcia funkcji heurystycznej $h(n)$. Wartość funkcji heurystycznej powinna określać przewidywaną drogę do węzła docelowego z bieżącego punktu. Pełny koszt $f(n)$ stanowi sumę dotychczasowego kosztu $g(n)$ oraz przewidywanego pozostałego kosztu.

\begin{gather}
 	f(n) = g(n) + h(n)
 	\label{eq_astar} 
\end{gather}
 gdzie:

 $g(n)$ - dotychczasowy koszt dotarcia do węzła $n$, dokładna odległość miedzy węzłem $n$ a węzłem początkowym

 $h(n)$ - heurystyka, przewidywana pozostała droga od węzła bieżącego do węzła docelowego

 $f(n)$ - oszacowanie pełnego kosztu ścieżki od węzła startowego do węzła docelowego prowadzącej przez węzeł $n$

 $n$ - bieżący węzeł, wierzchołek przeszukiwanego grafu

W każdym kroku przeszukiwany jest węzeł o najmniejszej wartości funkcji $f(n)$.
Dzięki takiemu podejściu najpierw sprawdzane są najbardziej "obiecujące" rozwiązania, co pozwala szybciej otrzymać wynik (w porównaniu do algorytmu Dijkstry).
Algorytm kończy działanie w momencie, gdy napotka węzeł będący węzłem docelowym.
Dla każdego odwiedzonego węzła zapamiętywane są wartości $g(n)$, $h(n)$ oraz węzeł będący rodzicem w celu późniejszego odnalezienia drogi powrotnej do węzła startowego po napotkaniu węzła docelowego (por. rys. \ref{fig:image_astar2}).

Algorytm zwraca optymalny wynik (najkrótszą możliwą ścieżkę), ale w pewnych warunkach (por. \ref{ch:astar_heurystyki}).

\begin{figure}
	\centering
	\includegraphics[width=10cm]{img/astar-t7}
	\caption{Ilustracja wyznaczania działania przez A*. Każdy odwiedzony węzeł wskazuje na swojego rodzica, co umożliwia późniejszą rekonstrukcję drogi. Źródło: \cite{astar2}}
	\label{fig:image_astar2}
\end{figure}

\subsection{Funkcja heurystyczna}
\label{ch:astar_heurystyki}
Od wyboru sposobu obliczania heurystyki zależy czas wykonywania algorytmu oraz optymalność wyznaczonego rozwiązania.

Funkcja heurystyczna $h(n)$ jest dopuszczalna, jeśli dla dowolnego węzła $n$ spełniony jest warunek
\begin{gather}
 	h(n) \leq h^*(n)
 	\label{eq_astar_heurystyka_admissible} 
\end{gather}
 gdzie:

 $h^*(n)$ - rzeczywisty koszt ścieżki od węzła $n$ do celu

Innymi słowy, heurystyka dopuszczalna to taka, która nigdy nie przeszacowuje pozostałej do przebycia drogi. Wtedy algorytm A* zwraca optymalną (najkrótszą) ścieżkę \cite{cooppath}.

A* opiera się na heurystyce, która "kieruje" przeszukiwaniem.
To od niej zależy, w kierunku których węzłów będzie podążało przeszukiwanie.
Źle wybrana heurystyka może prowadzić do zbędnego odwiedzania dodatkowych węzłów.

Poniżej przedstawiono najczęściej wykorzystywane heurystyki będące oszacowaniem odległości między przeszukiwanym węzłem $(x_n, y_n)$ a węzłem docelowym $(x_g, y_g)$ na dwuwymiarowej mapie.

\subsubsection{Heurystyka euklidesowa}
Heurystyka wykorzystująca metrykę euklidesową wyraża dokładną odległość po linii prostej. Wymaga to jednak częstego obliczania pierwiastków (lub stosowania {\it look-up table}).
\begin{gather}
 	h(n) = \sqrt{(x_n - x_g)^2 + (y_n - y_g)^2}
 	\label{eq_astar_heu_euc} 
\end{gather}

\subsubsection{Heurystyka Manhattan}
W przypadku, gdy agent może poruszać się po mapie jedynie poziomo lub pionowo (nie na ukos) wystarczająca okazuje się metryka Manhattan (metryka miejska):
\begin{gather}
 	h(n) = |x_n - x_g| + |y_n - y_g|
 	\label{eq_astar_heu_man} 
\end{gather}

\subsubsection{Heurystyka metryki maksimum}
Zastosowanie metryki maksimum (metryki Czebyszewa) może sprawdzić się np. dla niektórych figur szachowych:
\begin{gather}
 	h(n) = max(|x_n - x_g|, |y_n - y_g|)
 	\label{eq_astar_heu_czeb} 
\end{gather}

% \begin{figure}[H]
% 	\centering
% 	\includegraphics[width=8cm]{img/paclan1}
% 	\caption{Mapa z zaznaczeniem podziału na siatkę pól, zastosowanie heurystyki uwzględniającej zawijanie mapy po bokach, Źródło: własna implementacja gry Pac-Man}
% 	\label{fig:image_paclan1}
% \end{figure}

\subsubsection{Heurystyka zerowa}
Przyjęcie heurystyki równej $h(n) = 0$ sprawia, że algorytm A* sprowadza się do algorytmu Dijkstry.

\section{Metody ponownego planowania}
\subsection{Local Repair A*}
\label{ch:lra}
W algorytmie Local Repair A* (LRA*) każdy z agentów znajduje drogę do celu, używając algorytmu A*, ignorując pozostałe roboty oprócz ich obecnych sąsiadów. Roboty zaczynają podążać wyznaczonymi ścieżkami do momentu, aż kolizja z innym robotem jest nieuchronna (w lokalnym otoczeniu). Wtedy następuje ponowne przeliczenie drogi pozostałej do przebycia, z uwzględnieniem nowo napotkanej przeszkody.

Możliwe (i całkiem powszechne \cite{cooppath}) jest uzyskanie cykli (tych samych sekwencji ruchów powtarzających się w nieskończoność), dlatego zazwyczaj wprowadzane są pewne modyfikacje, aby rozwiązać ten problem. Jedną z możliwości jest zwiększanie wpływu losowego szumu na wartość heurystyki. Kiedy agenci zachowują się bardziej losowo, prawdopodobne jest, że wydostaną się z problematycznego położenia i spróbują podążać innymi ścieżkami.

Algorytm ten ma jednak sporo poważnych wad, które szczególnie ujawniają się w trudnych środowiskach z dużą liczbą przeszkód. Wydostanie się z zatłoczonego wąskiego gardła może trwać bardzo długo. Prowadzi to także do ponownego przeliczania trasy w prawie każdym kroku. Wciąż możliwe jest również odwiedzanie tych samych lokalizacji w wyniku zapętleń.

\subsection{Algorytm D*}
\label{ch:dstar}
D* ({\it Dynamic A* Search}) jest przyrostowym algorytmem przeszukiwania. Jest modyfikacją algorytmu A* pozwalającą na szybsze ponowne planowanie trasy w wyniku zmiany otoczenia (np. zajmowania wolnego pola przez innego robota). Wykorzystywany jest m.in. w nawigacji robota do określonego celu w nieznanym terenie. Początkowo robot planuje drogę na podstawie pewnych założeń (np. nieznany teren nie zawiera przeszkód). Podążając wyznaczoną ścieżką, robot odkrywa rzeczywisty stan mapy i jeśli to konieczne, wykonuje ponowne planowanie trasy na podstawie nowych informacji.
Często wykorzystywaną implementacją (z uwagi na zoptymalizowaną złożoność obliczeniową) jest wariant algorytmu {\it D* Lite} \cite{dstarlite}.

\subsection{D* Extra Lite}
\label{ch:dstarextralite}
Wartym uwagi jest także algorytm {\it D* Extra Lite} charakteryzujący się jeszcze korzystniejszą wydajnością niż {\it D* Lite}, co potwierdziły przeprowadzone obszerne testy w różnego rodzaju środowiskach (m.in. na mapach z gier komputerowych oraz w zawiłych labiryntach) \cite{przybylski}.

{\it D* Extra Lite} służy do przyrostowego planowania najkrótszej ścieżki w dwuwymiarowej przestrzeni bez dokładnej wiedzy o środowisku. Wykorzystanie wyników z poprzednich iteracji oraz wczesne odrzucanie pewnych węzłów z drzewa przeszukiwania znacznie skraca czas potrzebny do wykonania ponownego planowania.
{\it D* Extra Lite} jest nowatorskim algorytmem ogólnego przeznaczenia. Naturalnym jego zastosowaniem jest nawigacja robotów mobilnych \cite{przybylski}.

\section{Cooperative A*}
Cooperative A* jest algorytmem do rozwiązywania problemu kooperacyjnego znajdowania tras.
Metoda może być również nazywana czasoprzestrzennym algorytmem A* ({\it time-space A* search})
Zadanie planowania jest rozdzielone na serię pojedynczych poszukiwań dla poszczególnych agentów.
Pojedyncze poszukiwania są wykonywane w trójwymiarowej czasoprzestrzeni i biorą pod uwagę zaplanowane ścieżki przez pozostałych agentów.
Akcja wykonania postoju (pozostania w tym samym miejscu) jest uwzględniona w zbiorze akcji możliwych do wykonania.
Po przeliczeniu dróg dla każdego agenta, stany zajętości pól są zaznaczane w tablicy rezerwacji (ang. Reservation table).
Pozycje w tej tablicy są uważane jako pola nieprzejezdne i w efekcie są omijane podczas przeszukiwania przez późniejszych agentów \cite{cooppath}.

Należy zaznaczyć, że planowanie dla każdego agenta odbywa się sekwencyjnie według przydzielonych priorytetów.
Algorytm ten jest podatny na zmianę kolejności agentów. Odpowiedni dobór priorytetów może wpłynąć na wydajność algorytmu oraz jakość uzyskanego wyniku.

\subsection{Trzeci wymiar - czas}
Do rozwiązania problemu kooperacyjnego znajdowania dróg algorytm przeszukiwania potrzebuje mieć pełną wiedzę na temat przeszkód oraz jednostek na mapie.
Aby zapisać tą informację, potrzeba rozszerzyć mapę o trzeci wymiar - czas. 
Pierwotną mapę będziemy nazywać mapą przestrzenną, natomiast nową - czasoprzestrzenną mapą \cite{cooppath}.

Zagadnienie sprowadza się do przeszukiwania grafu, w którym każdy węzeł ma przypisane 3 wielkości: położenie x, położenie y oraz czas.
Podczas gdy zakres wielkości x i y jest znany i wynika z rozmiarów mapy oraz podziału jej wymiarów na dyskretne pola, to jednak określenie wymiaru czasu i jego granicy nie jest trywialnym zagadnieniem.
Wymiar czasu możemy również zdyskretyzować i przyjąć, że krok czasu jest okresem, jaki zajmuje robotowi przejście z jednego pola na sąsiednie (poziomo lub pionowo). Natomiast górną granicą czasu powinna być maksymalna liczba ruchów potrzebna do dotarcia do celu przez ostatniego robota. Wybór za małej liczby może spowodować, że algorytm nie zdąży znaleźć drogi dla niektórych agentów, z kolei zbyt duża granica kroków czasu znacząco wydłuża obliczenia. Rozwiązanie tego problemu zostało opisane w późniejszym podrozdziale \ref{ch:whca}.

Wprowadzenie trzeciego wymiaru wprowadza także konieczność zmian w doborze odpowiedniej heurystyki odpowiedzialnej za oszacowanie drogi pozostałej do celu.

\subsection{Tablica rezerwacji}
Tablica rezerwacji (ang. {\it Reservation Table}) reprezentuje współdzieloną wiedzę o zaplanowanych ścieżkach przez wszystkich agentów.
Jest to informacja o zajętości każdej z komórki na mapie w danym miejscu i określonym czasie \cite{cooppath}.
Jak tylko agent zaplanuje trasę, każda komórka odpowiadająca ścieżce zaznaczana jest jako zajęta w tablicy rezerwacji.

W prostej implementacji tablica rezerwacji jest trójwymiarową kostką (dwa wymiary przestrzenne i jeden wymiar czasu).
Każda komórka kostki, która jest przecinana przez zaplanowaną przez agenta ścieżkę, jest zaznaczana jako nieprzejezdna przez określony czas trwania. W ten sposób zapobiega to planowania kolizyjnych tras przez pozostałych agentów. (por. rys. \ref{fig:img_reservation-table})

\begin{figure}[H]
	\centering
	\includegraphics[width=13cm]{img/reservation-table}
	\caption{Dwie jednostki kooperacyjnie poszukujące tras. (A) Pierwsza jednostka znajduje ścieżkę i zaznacza ją w tablicy rezerwacji. (B) Druga jednostka znajduje ścieżkę, uwzględniając istniejące rezerwacje pól, również zaznaczając ją w tablicy rezerwacji. Źródło: \cite{cooppath}}
	\label{fig:img_reservation-table}
\end{figure}

Jeśli tylko niewielka część z całej tablicy rezerwacji będzie markowana jako zajęta, wydajniej jest zaimplementować ją jako tablicę typu {\it hash table}. Daje to zaletę oszczędności pamięci poprzez pamiętanie jedynie współrzędnych $(x, y, t)$ zajętych pól.

W ogólności poszczególni agenci mogą mieć różną prędkość lub rozmiary, zatem tablica rezerwacji musi mieć możliwość zaznaczenia dowolnego zajętego obszaru. Zostało to przedstawione na rysunku \ref{fig:img_reservation-table-3}.

\begin{figure}[H]
	\centering
	\includegraphics[width=6cm]{img/reservation-table-3}
	\caption{Tablica rezerwacji jest współdzielona między wszystkimi agentami. Jej rozmiar powinien być odpowiednio dopasowany do agentów o różnych prędkościach. Źródło: \cite{cooppath}}
	\label{fig:img_reservation-table-3}
\end{figure}

Niestety powyższy sposób wykorzystania tablicy rezerwacji w pewnych sytuacjach nie zapobiega zderzeniom czołowym jednostek zmierzających w przeciwnych kierunkach.
Jeśli jedna jednostka zarezerwowała komórki $(x, y, t)$ i $(x + 1, y, t + 1)$, nic nie stoi na przeszkodzie, aby kolejna jednostka mogła zarezerwować komórki $(x + 1, y, t)$ i $(x, y, t + 1)$. Ten problem może być rozwiązany poprzez zajmowanie (rezerwowanie) dwóch sąsiednich komórek w tym samym czasie $t$ podczas ruchu robota, a nie tylko jednej komórki.

\section{Hierarchical Cooperative A*}
\label{ch:hier_cooperative_a}
Metoda {\it Hierarchical Cooperative A*} (HCA*) wprowadza pewną modyfikację do algorytmu Cooperative A*. Modyfikacja ta dotyczy heurystyki opartej na abstrakcjach przestrzeni stanu \cite{cooppath}.
HCA* jest także jednym z przykładów rozproszonego podejścia do planowania tras.

W tym podejściu abstrakcja przestrzeni stanu oznacza zignorowanie wymiaru czasu, jak również tablicy rezerwacji.
Innymi słowy, abstrakcja jest prostą dwuwymiarową mapą z usuniętymi agentami. Abstrakcyjne odległości mogą być rozumiane jako dokładne oszacowania odległości do celu, ignorując potencjalne interakcje z innymi agentami. Jest to oczywiście dopuszczalna heurystyka (por. \ref{ch:astar_heurystyki}). Niedokładność heurystyki wynika jedynie z trudności związanych z interakcją z innymi agentami (jak bardzo agent musi zboczyć z pierwotnie zaplanowanej ścieżki w celu ominięcia innych agentów).

Do wyznaczenia heurystyki dla abstrakcyjnej przestrzeni stanu opisywane podejście wykorzystuje algorytm przeszukiwania {\it Reverse Resumable A*} (RRA*).
Algorytm ten wykonuje zmodyfikowane przeszukiwanie A* w odwrotnym kierunku. Przeszukiwanie zaczyna się w węźle docelowym agenta i kieruje się do początkowego położenia. Jednak zamiast kończyć w tym punkcie, przeszukiwanie jest kontynuowane do natrafienia na węzeł $N$, w którym znajduje się agent.

Algorytm HCA* jest więc taki, jak algorytm CA*, ale z bardziej wyszukaną heurystyką, która używa RRA* do obliczania abstrakcyjnych odległości na żądanie.

\section{Windowed Hierarchical Cooperative A*}
\label{ch:whca}
Problematyczne zagadnienie związane z wyżej wspomnianymi algorytmami jest takie, że kończą one działanie w momencie, gdy agent osiąga swój cel. Jeśli agent znajduje się już w miejscu docelowym, np. w wąskim korytarzu, to może on blokować części mapy dla innych agentów. W takiej sytuacji agenci powinni kontynuować kooperację z pozostałymi jednostkami, nawet po osiągnięciu swoich celów. Może to zostać zrealizowane np. poprzez usunięcie się z wąskiego gardła w celu przepuszczenia pozostałych agentów, a następnie powrót do docelowego punktu \cite{cooppath}.

Kolejny problem związany jest z wrażliwością na kolejność agentów (przydzielone priorytety). Chociaż czasem możliwy jest skuteczny, globalny przydział priorytetów \cite{latombe}, to jednak dobrym rozwiązaniem może być dynamiczne modyfikowanie kolejności agentów. Wtedy rozwiązania mogą zostać znalezione w tych przypadkach, w których zawiodło przydzielanie niezmiennych priorytetów \cite{cooppath}.

Rozwiązaniem powyższych kwestii jest zamknięcie algorytmu przeszukiwania w oknie czasowym.
Kooperacyjne planowanie jest ograniczone do ustalonej głębokości. Każdy agent szuka częściowej ścieżki do celu i zaczyna nią podążać. W regularnych okresach (np. gdy agent jest w połowie drogi) okno jest przesuwane dalej i wyznaczana jest nowa ścieżka.

Aby zapewnić, że agenci podążają do prawidłowych punktów docelowych, ograniczana jest tylko głębokość przeszukiwania kooperacyjnego (związanego z wieloma agentami), podczas gdy przeszukiwanie abstrakcyjnych odległości (heurystyki opisanej w podrozdziale \ref{ch:hier_cooperative_a}) odbywa się bez ograniczeń głębokości. Okno o rozmiarze $w$ może być rozumiane jako pośrednia abstrakcja, która jest równoważna wykonaniu $w$ kroków w rzeczywistym środowisku (z uwzględnieniem pozostałych agentów) a następnie wykonaniu pozostałych kroków zgodnie z abstrakcją (bez uwzględnienia innych agentów). Innymi słowy, pozostali agenci są jedynie rozważani dla $w$ pierwszych kroków (poprzez tablicę rezerwacji) a dla pozostałych kroków są ignorowani \cite{cooppath}.

Rozmiar okna jest wielkością ustalaną arbitralnie. Rozmiar okna powinien być przyjęty jako czas trwania najdłuższego przewidywanego wąskiego gardła (zatoru).
Dobrą praktyką jest przyjęcie wartości równej liczbie agentów na mapie, gdyż to właśnie z ich powodów mogą wystąpić ewentualne zmiany w zaplanowanej trasie.

\subsubsection{Porównanie HCA* i WHCA*}
Algorytm HCA* wybiera ustaloną kolejność agentów i planuje trasy dla każdego agenta po kolei, unikając kolizji z poprzednio wyznaczonymi ścieżkami. 
Natomiast użycie przeszukiwania z przesuwanym oknem w WHCA* poprawia skuteczność algorytmu oraz przyspiesza proces wyznaczania rozwiązania \cite{completealgo_standley}.

Fakt wykonywania przeszukiwania w oknie oznacza, że planowanie algorytmem WHCA* wykonywane jest zawsze z ustaloną liczbą kroków w przyszłości i wybierany jest najbardziej obiecujący węzeł na granicy tego okna \cite{rtcooppathfinding}. W metodach Hierarchical Cooperative A* oraz Cooperative A* wybór granicy wymiaru czasu (głębokości przeszukiwania w liczbie kroków) stanowi balans pomiędzy wydajnością a zupełnością algorytmu.

W obu podejściach HCA* i WHCA* zastosowano dodatkowy algorytm przeszukiwania wstecz (RRA*) wspomagający heurystykę. Służy on wyznaczeniu dokładnej odległości z węzła do celu, pomijając wpływ innych agentów. Jest to często heurystyka wysokiej jakości (prawie idealne oszacowanie), gorzej sprawdza się jedynie w środowiskach z dużą ilością wąskich gardeł i zatorów \cite{rtcooppathfinding}.

Chociaż takie przeszukiwanie wstecz prowadzi do początkowego wykonania większej ilości obliczeń (wykonanie pełnego przeszukania A* z punktu docelowego do punktu startowego, jak również do innych węzłów), to jednak koszt obliczeniowy w kolejnych krokach jest już zdecydowanie niższy \cite{rtcooppathfinding}.
