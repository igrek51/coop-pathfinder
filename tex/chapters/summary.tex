\chapter{Podsumowanie}
\label{ch:podsumowanie}

$TODO$
Większość algorytmów Cooperative Pathfinding opiera się o A*

Algorytmy wprowadzają ograniczenie (błędne założenie), że ruchy trwają tyle samo. Można robić inaczej: podzielić dyskretnie i zaznaczać zajętość pól w wielu kratkach - ale wtedy będzie więcej obliczeń.

Windowed Hierarchical Cooperative A*:
• Cooperative A*
• Hierarchical Heuristic
• Windowed cooperation

The cooperative pathfinding methods are more successful
and find higher quality routes than A* with Local Repair.
Unfortunately, the basic CA* algorithm is costly to compute,

The size of the window has a significant effect on the suc-
cess and performance of the algorithm. With a large win-
dow, WHCA* behaves more like HCA* and the initialisa-
tion time increases. If the window is small, WHCA* be-
haves more like Local Repair A*. The success rate goes
down and the path length increases. The window size pa-
rameter thus provides a spectrum between Local Repair A*
and HCA*. An intermediate choice appears to give the most robust overall performance.
In general, the window size should be set to the duration
of the longest predicted bottleneck. In Real-Time Strategy
games groups of units are often moved together towards a
common destination. In this case the maximum bottleneck
time with cooperative pathfinding (ignoring units in other
groups) is the number of units in the group. If the window
size is lower than this threshold, bottleneck situations may
not be resolved well. If the window size is higher, then some
redundant search will be performed.

Local Repair A* may be an adequate solution for simple en-
vironments with few bottlenecks and few agents. With more
difficult environments, Local Repair A* is inadequate and is
significantly outperformed by Cooperative A* algorithms.

By introducing Hierarchical A* to improve the heuristic and
using windowing to shorten the search, a robust and efficient
algorithm, WHCA*, was developed. WHCA* finds more
successful routes than Local Repair A*, with shorter paths
and fewer cycles.

Although this research was restricted to grid environ-
ments, the algorithms presented here apply equally to more
general pathfinding domains. Any continuous environment
or navigational mesh can be used, so long as each agent’s
route can be planned by discrete motion elements. The grid-
based reservation table is generally applicable, but reserving
mesh intersections may be more appropriate in some cases.
Finally, the cooperative algorithms may be applied in dy-
namic environments, so long as an agent’s route is recom-
puted whenever invalidated by a change to the map.


Cooperative pathfinding is a general technique for coordinating the paths of many units.
It is appropriate whenever there are many units on the same side who are able to
communicate their paths. By planning ahead in time as well as space, units can get out of
each other’s way and avoid any conflicting routes.

Many of the usual enhancements to spatial A* can also be applied to space-time A*.
Moreover, the time dimension gives a whole new set of opportunities for pathfinding
algorithms to explore.
