% \section{Pytania}
\begin{itemize}
	\item demo: LRA*, autorski WHCA* + dynamiczne priorytety + dynamiczne okno czasowe
	\item filmiki, publikacja na GitHub, link zamieszczony w pracy, YT
	\item rozdział Opracowanie i implementacja algorytmów - rdzeń logiki (3 metody)
	\begin{itemize}
		\item (X) generator map (własności), (X) A*, (X) LRA*
		\item własna implementacja WHCA*,
		\item autorska metoda dynamicznego przydziału priorytetów + okna czasowego
	\end{itemize}
	\item (X) Oprogramowanie symulacyjne - software development
	\begin{itemize}
		\item architektura MVP - testy skuteczności
		\item wykorzystane technologie
		\item TDD, jUnit
		\item wielowątkowość
	\end{itemize}
	\item rozdział Wyniki testów
	\begin{itemize}
		\item obszerne testy automatyczne: skuteczności i wydajności metod
		\item 3 środowiska testowe: 11x11, 5 robotów; 11x11, 10 robotów; 35x35, 5 robotów
		\begin{itemize}
			\item charakterystyka map - jak często nie wystarcza zwykły A*, bo występują kolizje?
			\item porównanie skuteczności LRA* z WHCA* (z promocją priorytetów) w tych samych warunkach
			\item porównanie samego WHCA* z/bez autorską promocją priorytetów
			\item porównanie WHCA* z promocją priorytetów + rozszerzanie okna czasowego
			\item liczba kroków potrzebna do rozwiązania (LRA*, WHCA*, WHCA*+priorytety): histogram
		\end{itemize}
		\item odróżnienie testów wydajności od testów poprawności
	\end{itemize}
	\item aplikacja 100\%, 75 / 100 PDF
	\item przedłużenie - wniosek
	\item wysyłać całość, czy rozdziałami?
\end{itemize}