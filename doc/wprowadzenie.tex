\chapter{Wprowadzenie} % (fold)
\label{cha:wprowadzenie}

Celem pracy jest zaprojektowanie i stworzenie systemu do monitorowania zużycia mediów komunalnych takich jak woda czy gaz.
System ten ma być przeznaczony do współpracy z licznikami mediów wyposażonymi w interfejs komunikacyjny M-Bus.
Poza przedstawieniem aktualnych stanów liczników, system powinien pozwalać na wizualizację historycznego zużycia mediów.

Projektowany system ma być przeznaczony do użytku w budynkach mieszkalnych.
Nie można zatem zakładać, że użytkownikami systemu będą osoby znające zagadnienia techniczne związane z protokołem komunikacyjnym M-Bus.
Należy więc stworzyć prosty i intuicyjny interfejs graficzny, na którym będą prezentowane tylko niezbędne informacje.

Dane o zużyciu powinny być dostępne poprzez sieci LAN oraz WiFi, jako że obecnie są one bardzo popularne i używane w wielu domach.
Najprostszym sposobem na osiągnięcie tego celu jest udostępnianie ich za pomocą strony WWW.
Umożliwi to dostęp do danych zarówno z komputerów typu PC, laptopów jaki i tabletów czy smartfonów.

System powinien również wykrywać błędy, mogące pojawić się w trakcie użytkowania, takie jak brak połączenia z podsystemami i urządzeniami składowymi, odłączenia lub uszkodzenia liczników.
Informacje o wykrytych błędach powinny być dostępne z poziomu interfejsu graficznego w postaci strony WWW.

Dodatkową funkcjonalnością jaką powinien zapewniać system jest analiza profilów zużycia mediów dla poszczególnych liczników należących do systemu.
Aby stworzyć profile zużycia wymagane jest możliwie częste odczytywanie liczników i przechowywanie wszystkich historycznych danych.
Profile powinny być ciągle uaktualniane, aby odzwierciedlić jak najlepiej charakter zmian zużycia mediów.
Dzięki wyznaczonemu profilowi możliwe będzie sprawdzenie, czy aktualne zużycie danego medium znaczącą odbiega od przewidywanej wartości lub czy nie.
Przypadki wyraźnej niezgodności z profilami zużycia mediów mogą oznaczać uszkodzenia infrastruktury doprowadzającej lub uszkodzenia liczników i powinny być raportowane podobnie jak błędy.

Projektowany system może być stosowany do celów rozliczeniowych oraz kontrolnych, np. w mieszkaniach wynajmowanych, gdzie właściciel nie mam możliwości bezpośredniego sprawdzenia stanów liczników.

Praca skupia się na badaniu współpracy systemu z licznikami wody.
Jednak system powinien działać poprawnie z 250 licznikami dowolnych mediów komunalnych, niezależnie od ich modelu czy producenta.
% chapter wprowadzenie (end)
