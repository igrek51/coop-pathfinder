\noindent {\Large\textbf{Streszczenie}}

Niniejsza praca na celu zaprojektowanie oraz stworzenie systemu do monitorowania zużycia mediów komunalnych.
System ma współpracować z licznikami mediów wyposażonymi w interfejs komunikacyjny M-Bus.
System ma przedstawiać aktualne stany liczników oraz wizualizować historyczne zużycie mediów.
Dodatkowym zadaniem systemu jest tworzenie i analiza profili zużycia mediów komunalnych.
Analiza profilu ma na celu określenie czy aktualna wartość zużycia odbiega, w sposób znaczący, od wartości przewidywanej.
System ma informować o zaistniałych odejściach od wartości oczekiwanej oraz innych błędach, które mogą świadczyć o uszkodzeniu liczników lub infrastruktury doprowadzającej media.
Praca skupia się na współpracy systemu z licznikami wody.

Do zadań projektu należało stworzenie oprogramowania zajmującego się odczytem liczników zgodnie z protokołem M-Bus oraz interfejsu użytkownika udostępniającego, w przystępny sposób, zebrane dane.

W pierwszej części pracy został opisany protokół M-Bus z podziałem na warstwę fizyczną, łącza danych, aplikacji oraz zarządzania.
Przedstawione zostały tryby transmisji danych, formaty ramek telekomunikacyjnych, struktury rekordów z danymi oraz sposób zwiększenia obsługiwanej liczby urządzeń w sieci, dzięki rozszerzonemu adresowaniu.
Opis jest zgodny z normą PN-EN 13757-3:2013.

Ponieważ działanie systemu miało zostać przetestowane na licznikach wody, kolejna część pracy zawiera opis najczęściej używanych wodomierzy.
Są to wodomierze wirnikowe oraz komorowe służące często do wzorcowania tych pierwszych.
Przedstawione zostały ich schematy oraz zasady działania.

Dalej znajduje się opis dostępnego rozwiązania o podobnym zastosowaniu ZENNER Bus-System.
Zaprezentowane zostały urządzenia komunikacyjne współpracujące z magistralą M-Bus.

Kolejna część pracy przedstawia zbiór wymagań dotyczących działania systemu oraz sposób ich implementacji w wyniku, której powstały dwa podprojekty odpowiedzialne odpowiednio za odczytywanie danych z liczników oraz wyświetlanie tych danych na stronie WWW.
Wyznaczona zostaje tu wartości interwału odczytów.
Następnie zostaje zaprezentowany główny algorytm  programu odczytującego wraz z opisem klas występujących projekcie.

Ostatnia część zawiera opis testów systemu, mających na celu sprawdzenie zgodności z postawionymi wcześniej wymaganiami.
Testy zostały zrealizowane przy pomocy dostępnych wodomierzy oraz specjalnie do tego stworzonej aplikacji symulującej licznik.

\pagebreak

\noindent {\Large\textbf{Abstract}}


\pagebreak